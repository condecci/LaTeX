%   EN EL SIGUIENTE LINK SE ENCUENTRA EL ARCHIVO EN OVERLEAF PARA MOSTRAR EL RESULTADO DEL PRESENTE CÓDIGO
%   https://www.overleaf.com/7392513792gsmkscvzpfzw#a6a8c8

\documentclass{article}
\usepackage{graphicx} % Required for inserting images
\usepackage[spanish]{babel} %para definir el idioma
% Para especificar los márgenes
\usepackage[letterpaper,top=2cm,left=1.7cm,right=1.7cm,marginparwidth=1.5cm]{geometry}

%librerias para insertar códigos
\usepackage{listings}
\usepackage{mdframed}
\usepackage{caption}

\usepackage{xcolor}
\definecolor{grisClaro}{RGB}{232, 232, 232}

\title{Inserción de códigos del lenguaje VHDL}
\author{Mariana Conde}
\date{}

%   DEFINIR UN CÓDIGO DE PROGRAMACIÓN
\lstdefinelanguage{VHDL}{
%   ESPECIFICAR LAS PALABRAS RESERVADAS
   morekeywords={
    architecture, entity, is, port, in, out, signal, process, begin, end, if, then, else, when, others, case, report, severity, library, use, all, wait, for, generic,type,signal,rising_edge, attribute, elsif, end case
  },
    sensitive=true,
    morecomment=[l]--,     % COMENTARIO DE UNA LÍNEA
    morestring=[b]",       % COMILLAS DOBLES PARA CADENAS DE TEXTO
}
%   CONFIGURACIÓN DEL CÓDIGO
\lstset{
  language=VHDL,
  basicstyle=\ttfamily\footnotesize, % ESTILO DE FUENTE (monoespaciada)
  keywordstyle=\color{blue}\bfseries, % PALABRAS CLAVE EN AZUL Y NEGRITA
  commentstyle=\color{gray}, % COMENTARIOS EN GRIS
  stringstyle=\color{red}, % CADENAS DE TEXTO EN ROJO
  numbers=left, % MOSTRAR NÚMEROS DE LÍNEA A LA IZQUIERDA
  numberstyle=\tiny\color{gray}, % ESTILO DE LOS NÚMEROS DE LÍNEA
  stepnumber=1, % NUMERAR CADA LÍNEA
  breaklines=true, % DIVIDIR LAS LÍNEAS LARGAS
  tabsize=2, % TAMAÑO DE TABULACIÓN
  captionpos = t
}
%   PERSONALIZACIÓN DEL TÍTULO DEL CÓDIGO
\DeclareCaptionFont{customfont}{\small}
\captionsetup[lstlisting]{font=customfont, labelformat=simple}
\renewcommand{\lstlistingname}{Código}

%   FUNCIÓN PARA ENCUADRAR EL CÓDIGO
%   SÓLO SE COLOCAN LAS LÍNEAS SUPERIOR E INFERIOR
\mdfdefinestyle{topbottomframe}{
    topline = true,
    bottomline = true,
    leftline = false,
    rightline = false
}
%   PERSONALIZACIÓN DEL TÍTULO DEL CÓDIGO
\DeclareCaptionFont{customfont}{\small}
\captionsetup[lstlisting]{font=customfont, labelformat=simple}
\renewcommand{\lstlistingname}{Código}


\begin{document}

\maketitle

\begin{mdframed}[style=topbottomframe]
\begin{lstlisting}[caption={Divisor de frecuencia},label={lst:div_freq}]
library ieee;
use ieee.std_logic_1164.all;

entity DIV_FREQ is
generic(limit: integer := 249999);
port(CLK:	in std_logic;
	  DIVF:	buffer std_logic);
end DIV_FREQ;

architecture clk of DIV_FREQ is
begin
    process(CLK)
    begin
        if rising_edge(CLK) then
            if AUX = limit then
               AUX <= 0;
               AUX <= NOT AUX;
            else
					AUX <= AUX + 1;
            end if;
        end if;
    end process;
end clk;
\end{lstlisting}
\end{mdframed}


\begin{mdframed}[style=topbottomframe]
\begin{lstlisting}[caption={Máquina de estados},label={lst:state_machine}]
library IEEE;
use IEEE.STD_LOGIC_1164.ALL;

entity PuertadeAuto is
	Port(clk:               in  STD_LOGIC;
         sw1,sw2,lsw1,lsw2: in std_logic;
         ENA1,ENA2:         out std_logic);
end PuertadeAuto;

architecture estructura of PuertadeAuto is
type estados is (inicio,e1,baja,guarda1,sube,guarda2,paro1,paro2);
signal presente: estados;
signal div: std_logic;
signal SW_1,SW_2,LSW_1,LSW_2: std_logic;
attribute enum_encoding: string;
attribute enum_encoding of estados: type is "compact";

component DIV_FREQ is
	port( clk:	in std_logic;
	   DIVF:	out std_logic);
end component;
begin
reloj: DIV_FREQ port map(clk => clk, DIVF =>div);
SW_1 <= SW1;
SW_2 <= SW2;
LSW_1 <= LSW1;
LSW_2 <= LSW2;
process(div)
begin
	if  rising_edge(div) then
		case (presente) is
		when inicio => 
			ENA1 <= '0';
			ENA2 <= '0';
			presente <= e1;
		when e1 =>
			ENA1 <= '0'; ENA2 <= '0';
			if (SW_1 = '1' and SW_2 = '0' and LSW_1 = '0' and LSW_2 = '0')   then
				presente <= baja;
			elsif (SW_1 = '0' and SW_2 = '1' and LSW_1 = '0' and LSW_2 = '0') then
				presente <= sube;
			elsif (SW_1 = '0' and SW_2 = '1' and LSW_1 = '1' and LSW_2 = '0') then
				presente <= sube;
			elsif (SW_1 = '1' and SW_2 = '0' and LSW_1 = '0' and LSW_2 = '1') then
				presente <= baja;
			end if;
       when baja =>
    			ENA1 <= '1'; ENA2 <= '0';
    			presente <= guarda1;
    		when guarda1 =>
    			ENA1 <= '1'; ENA2 <= '0';
    			if(LSW_1 = '1' and LSW_2 = '0') then
    				presente <= paro1;
    			else
    				presente <= inicio;
    			end if;
    		when sube =>
    			ENA1 <= '0'; ENA2 <= '1';
    			presente <= guarda2;
    		when guarda2 =>
    			if(LSW_1 = '0' and LSW_2 = '1') then
    				presente <= paro2;
    			else
    				presente <= inicio;
    			end if;
    		when paro1 =>
    			ENA1 <= '0'; ENA2 <= '0';
    			presente <= inicio;
    		when paro2 =>
    			ENA1 <= '0'; ENA2 <= '0';
    			presente <= inicio;
    		when others =>
    			ENA1 <= '0';
    			ENA2 <= '0';
    			presente <= e1;
		end case;
	end if;
end process;
end estructura;
\end{lstlisting}
\end{mdframed}

\end{document}
