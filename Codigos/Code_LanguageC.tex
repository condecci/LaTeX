%   EN EL SIGUIENTE LINK SE ENCUENTRA EL ARCHIVO EN OVERLEAF PARA MOSTRAR EL RESULTADO DEL PRESENTE CÓDIGO
%   https://www.overleaf.com/project/6736d782fb1ad1644ce4260a

\documentclass{article}
\usepackage{graphicx} % Required for inserting images
\usepackage[spanish]{babel} %para definir el idioma
% Para especificar los márgenes
\usepackage[letterpaper,top=2cm,left=1.7cm,right=1.7cm,marginparwidth=1.5cm]{geometry}

%librerias para insertar códigos
\usepackage{listings}
\usepackage{mdframed}
\usepackage{caption}

\usepackage{xcolor}
\definecolor{grisClaro}{RGB}{232,232,232} %definir un nuevo color
\definecolor{StringColor}{RGB}{9,159,70}
\definecolor{CommentColor}{RGB}{162,162,162}
\definecolor{libreria}{RGB}{159,9,154}
\definecolor{tareas}{RGB}{0,32,233}

\title{Inserción de códigos de lenguaje C}
\author{Mariana Conde}
\date{}

%   DEFINIR UN CÓDIGO DE PROGRAMACIÓN
\lstdefinelanguage{CStyle}{
%   ESPECIFICAR LAS PALABRAS RESERVADAS
%   PARA ESTE CÓDIGO EN PARTICULAR SE UTILIZARON 3 morekeywords PORQUE CADA PALABRA RESERVADA CONTENIDA EN ÉSTOS TIENE UN COLOR DIFERENTE
    morekeywords = {
        volatile, static, NULL, true, const, false
    },
    morekeywords = [2]{
        include, float, char, void, bool, for, while, int, if, else, return
    },
    morekeywords = [3]{
        xTaskCreate, uin32_t, mutex_init, vTaskStartScheduler, stdio_init_all, gpio_init, gpio_set_dir, gpio_put, adc_init, adc_set_temp_sensor_enabled, adc_select_input, printf, mutex_enter_blocking, mutex_exit, vTaskDelay, pdMS_TO_TICKS, uint16_t, adc_read, getchar, putchar, readInt, setup
    },
    sensitive=true,
    morecomment=[l]//,     % COMENTARIO DE UNA LÍNEA
    morecomment=[l]/*,     % COMENTARIO DE MÁS DE UNA LÍNEA
    morestring=[b]",       % COMILLAS DOBLES PARA CADENAS DE TEXTO
}
%   CONFIGURACIÓN DEL CÓDIGO
\lstset{
  language=CStyle,
  basicstyle=\ttfamily\footnotesize, % ESTILO DE FUENTE (monoespaciada)
  keywordstyle=\color{red}\bfseries, % PALABRAS CLAVE EN COLOR ROJO
  keywordstyle=[2]\color{libreria},  % PALABRAS CLAVE EN EL COLO DEFINIDO "LIBRERIA"
  keywordstyle=[3]\color{tareas},   % PALABRAS CLAVE EN EL COLO DEFINIDO "TAREAS"
  commentstyle=\color{CommentColor}, % COMENTARIOS EN EL COLOR DEFINIDO "CommentColor"
  stringstyle=\color{StringColor}, % CADENAS DE TEXTO EN EL COLOR DEFINIDO "StringColor"
  numbers=left, % MOSTRAR NÚMEROS DE LÍNEA A LA IZQUIERDA
  numberstyle=\tiny\color{gray}, % ESTILO DE LOS NÚMEROS DE LÍNEA EN COLOR GRIS
  stepnumber=1, % ENUMERAR CADA LÍNEA
  breaklines=true, % DIVIDIR LAS LÍNEAS LARGAS
  tabsize=2, % TAMAÑO DE TABULACIÓN
  captionpos = t
}
%   FUNCIÓN PARA ENCUADRAR EL CÓDIGO
%   SÓLO SE COLOCAN LAS LÍNEAS SUPERIOR E INFERIOR
\mdfdefinestyle{topbottomframe}{
    topline = true,
    bottomline = true,
    leftline = false,
    rightline = false
}
%   PERSONALIZACIÓN DEL TÍTULO DEL CÓDIGO
\DeclareCaptionFont{customfont}{\small}
\captionsetup[lstlisting]{font=customfont, labelformat=simple}
\renewcommand{\lstlistingname}{Código}


\begin{document}

\maketitle

\begin{mdframed}[style=topbottomframe]
\begin{lstlisting}[caption={\textcolor{gray}{\textbf{main.c - Temperatura obtenida}}},label={lst:temperatura_blink}]
#include <hardware/adc.h>
void main(void){
    xTaskCreate( tempTask,                 /* The function that implements the task. */
            "temp",                   /* The text name assigned to the task - for debug only as it is not used by the kernel. */
            configMINIMAL_STACK_SIZE,  /* The size of the stack to allocate to the task. */
            NULL,                      /* The parameter passed to the task - not used in this case. */
            1,                         /* The priority assigned to the task. */
            NULL );  
}
static void tempTask(void *param){
const float K = 3.3f/(1<<12);
  while(true){
    uint16_t adc = adc_read();
    float voltage = (adc * K);
    float tempC = 27-(voltage-0.706)*(1/0.001721);
    float tempF = (tempC * 9/5) + 32;
    printf("Temperatura %0.2f \circ C\n",tempC);
    printf("Temperatura %0.2f \circ F\n",tempF);
    vTaskDelay(pdMS_TO_TICKS(1000));
  }
}
\end{lstlisting}
\end{mdframed}


\begin{mdframed}[style=topbottomframe]
\begin{lstlisting}[caption={\textcolor{gray}{\textbf{main.c - Atenuación del led centinela}}},label={lst:atenuacion_blink}]
static void blinkTask( void *param ){
    uint32_t brillo;
    const uint32_t periodo = 20;
    printf("Starting blink.\n");
    while(true){
        mutex_enter_blocking(&mutex);
        brillo = svDelay;
        mutex_exit(&mutex);
        float encendido = (brillo * periodo) / 100;
        float apagado = periodo - encendido;
        gpio_put(PICO_DEFAULT_LED_PIN,1);
        vTaskDelay(pdMS_TO_TICKS(encendido));        
        gpio_put(PICO_DEFAULT_LED_PIN,0);
        vTaskDelay(pdMS_TO_TICKS(apagado));        
    }
}
static void usbTask( void *param ){
    uint32_t delay;
    vTaskDelay(pdMS_TO_TICKS(400));
    while(true){
        printf("Nivel de brillo deseado: ");
        if(!readInt(&delay) || delay > 100){
            printf("\nValor invalido\n");
            continue;
        }
        mutex_enter_blocking(&mutex);
        svDelay = delay;
        mutex_exit(&mutex);
        printf("\nDelay set to %d\n", delay);
    }
}
\end{lstlisting}
\end{mdframed}

\end{document}
